%===============================================================================
% Template Name:      Shanghai University Presentation template
% Template URI:       https://github.com/hnyang1107/SHUbeamer
% Description:        Starter Presentation template for  
%                     Department of Mathematics, 
%                     Shanghai University
% Author:             Yang Haonan
% License:            MIT License
% License URI:        http://opensource.org/licenses/MIT
%===============================================================================
% !TEX program = xelatex
\documentclass[compress,serif,11pt]{beamer}

%=================================================
% theme and color
%=================================================
\usetheme{Warsaw} %Themes http://www.hartwork.org/beamer-theme-matrix/
\definecolor{colorA}{RGB}{21, 73, 154}
\definecolor{colorB}{RGB}{140, 151, 154}
%\definecolor{secinhead}{RGB}{249,196,95}
%\definecolor{titlebg}{RGB}{51,51,51}
\setbeamercolor{structure}{fg=colorA,bg=colorB}
%\setbeamercolor{secsubsec}{fg=secinhead,bg=black}
%\setbeamercolor{frametitle}{fg=secinhead,bg=titlebg}

%=================================================
% packages and new commands
%=================================================
\usepackage[ruled, linesnumbered, vlined]{algorithm2e}
\usepackage{epsfig, subfigure, amssymb, multirow, multicol, algorithmic, amsmath}
\usepackage{graphicx}
\usepackage{ctex} % 支持中文
\newcommand*{\superscript}[1]{\ensuremath{^{\rm #1}}}
\newcommand*{\subscript}[1]{\ensuremath{_{\rm #1}}}

%==================================================
% 在 section 页面分两列作为过渡
%==================================================
\AtBeginSection[]
{
    \begin{frame}[plain]
    \large
    \begin{multicols}{2}
    \tableofcontents[currentsection]
    \end{multicols}
    % \tableofcontents[currentsection,hideallsubsections]
    \end{frame}
}

%=================================================
% thesis details (preamble)
%=================================================
\title[{\sc The short title of your thesis } \hspace{0.8cm} \insertframenumber/\inserttotalframenumber]{{\sc The title of your thesis or dissertation should be typed here }}
\author[Presentation to some students --- {\sc Feb 10\superscript{th}, 2018}]{{Student name}}
\date{10 Feb 2018}
\institute{Department of Mathematics \\ Shanghai University}


%=================================================
% start presentation
%=================================================
\begin{document}

%========================
% title page
%========================
\begin{frame}[plain]
  \begin{center}
    \vspace{-0.1cm}
    \includegraphics[scale=0.14]{shu_logo.jpg}
  \end{center}
  \titlepage
\end{frame}

%========================
% content page
%========================
\begin{frame}
\frametitle{Contents}
\begin{multicols}{2}
  \tableofcontents
\end{multicols}
\end{frame}

%========================
% your slides:
%========================
\input{slides/overview}

%========================
% example slides:
%========================
\input{examples/lists}
\input{examples/columns}
\input{examples/figures}
\input{examples/description}
\section{Tables}
% --------------------------------------------------- Slide --
%\subsection{Tables}
\label{tables}
\begin{frame}\frametitle{Tables}
  \begin{table}
    \begin{tabular}{l | c | c | c | c }
      Competitor Name & Swim & Cycle & Run & Total \\
      \hline \hline
      John T & 13:04 & 24:15 & 18:34 & 55:53 \\ 
      Norman P & 8:00 & 22:45 & 23:02 & 53:47\\
      Alex K & 14:00 & 28:00 & n/a & n/a\\
      Sarah H & 9:22 & 21:10 & 24:03 & 54:35 
    \end{tabular}
    \caption{Triathlon results}
  \end{table}
  aaa\cite{zhongbao2003reply}
\end{frame}
\input{examples/blocks}
\input{examples/definition}
\input{examples/example}
\input{examples/theorem}
\input{examples/hyperlinks}
\input{examples/algorithms}

%========================
% bibliography
%========================
%%%%%%%%%%%%%%%%%%
%
% bibliography
%
%%%%%%%%%%%%%%%%%%

\begin{frame} \frametitle{References}
\begin{thebibliography}{xx}\footnotesize

\bibitem{FominFVGrandoniFKratschD2009} {\sc Fomin FV, Grandoni F \& Kratsch D}, 2009, {\em A note on the complexity of minimum dominating set }, Journal of Discrete Algorithms, {\bf{4(2)}}, pp.\ 209--214.
cite here\footfullcite{zhongbao2003reply}
\bibitem{critical} {\sc Grobler PJP \& Mynhardt CM}, 2009, {\em Secure domination critical graphs}, Discrete Mathematics, {\bf 309}, pp.~5820--5827.

\bibitem{VRB}{{\sc Van Rooij JMM \& Bodlaender HL}, 2011, {\em Exact algorithms for dominating set}, Discrete Applied Mathematics, {\bf 159}, pp.\ 2147--2164.}

\end{thebibliography}
\end{frame}

%=================================================
% end presentation
%=================================================
\end{document}